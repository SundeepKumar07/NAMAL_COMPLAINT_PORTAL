\documentclass[12pt,a4paper]{article}
\usepackage[utf8]{inputenc}
\usepackage[margin=1in]{geometry}
\usepackage{graphicx}
\usepackage{hyperref}
\usepackage{tabularx}
\usepackage{booktabs}
\usepackage{enumitem}

\title{\textbf{Namal Complaint Management System}}
\author{}
\date{}

\begin{document}

\maketitle

\section*{Development Team}
Dev Wizard

\subsection*{Team Members}

\textbf{Member 1}

\textbf{Name:} Sundeep Kumar

\textbf{Roll no:} NUM-BSCS-2024-75

\textbf{Email:} \href{mailto:bscs24f75@namal.edu.pk}{bscs24f75@namal.edu.pk}

\vspace{0.3cm}

\textbf{Member 2}

\textbf{Name:} Rehan Tariq

\textbf{Roll no:} NUM-BSCS-2024-66

\textbf{Email:} \href{mailto:bscs24f66@namal.edu.pk}{bscs24f66@namal.edu.pk}

\vspace{0.3cm}

\textbf{Member 3}

\textbf{Name:} Aimen Shafiq

\textbf{Roll no:} NUM-BSCS-2024-04

\textbf{Email:} \href{mailto:bscs24f04@namal.edu.pk}{bscs24f04@namal.edu.pk}

\vspace{0.5cm}

\textbf{Subject:} CSC-225 -- Software Engineering

\textbf{Department of Computer Science, Namal University, Mianwali}

\textbf{Submission Date:} Sunday 3\textsuperscript{rd} November 2025

\newpage

\section{Requirement Provider Agreement}

This Agreement is made and entered into on this day, [Insert Date], by and between:

Party 1: Sundeep Kumar\\
Party 2: Kamran Younis (RP)

\subsection{Purpose of the Agreement}

The purpose of this Agreement is to define the roles, responsibilities, milestones, and terms for the successful completion of the [Project Name] using the Kanban framework. The project aims to achieve the agreed objectives in a structured and timely manner while maintaining clear communication between both parties.

\subsection{Scope of Work}

The Parties agree to collaboratively work on the project according to the following milestones:

\textbf{Milestone 1: Documentation}

\begin{itemize}
    \item Prepare all necessary documentation, including project objectives, methodology, tools, technologies, and references.
    \item Review and finalize the project proposal for approval.
\end{itemize}

\textbf{Milestone 2: Planning \& Brainstorming}

\begin{itemize}
    \item Analyze requirements and plan the implementation strategy.
    \item Identify tasks, assign priorities, and prepare the Kanban board for workflow management.
    \item Decide on subtasks and timelines for development.
\end{itemize}

\textbf{Milestone 3: Development \& Implementation}

\begin{itemize}
    \item Execute tasks as per the Kanban board.
    \item Continuous review and progress updates between Parties.
    \item Identify and resolve any roadblocks during development.
\end{itemize}

\textbf{Milestone 4: Testing \& Delivery}

\begin{itemize}
    \item Conduct final testing of the deliverables.
    \item Prepare project reports and final documentation.
    \item Deliver the completed project and ensure mutual approval.
\end{itemize}

\subsection{Roles and Responsibilities}

\begin{itemize}
    \item \textbf{Sundeep Kumar:} Responsible for overall project management, task execution, documentation, and technical implementation.
    \item \textbf{RP:} Responsible for reviewing progress, providing feedback, assisting in planning, and approving milestones.
\end{itemize}

Both Parties agree to maintain transparency, provide timely updates, and actively participate in decision-making throughout the project lifecycle.

\subsection{Workflow Methodology}

The Parties agree to follow the Kanban framework, which includes:

\begin{itemize}
    \item Breaking the project into manageable tasks and subtasks.
    \item Visualizing the workflow on a Kanban board.
    \item Continuously monitoring and updating task status (To Do, In Progress, Done).
    \item Conducting periodic review sessions to reassess priorities and reassign tasks if necessary.
\end{itemize}

\subsection{Deliverables}

\begin{itemize}
    \item Completed documentation, including project proposal and final report.
    \item Fully developed and tested project as per the agreed specifications.
    \item Progress updates and \textbf{milestone reports at the end of each milestone.}
\end{itemize}

\subsection{Terms and Conditions}

\begin{itemize}
    \item Any changes in scope or additional tasks require mutual consent and documentation.
    \item Both Parties agree to maintain confidentiality of project details.
    \item The project timeline is tentative and subject to adjustments based on project complexity and mutual agreement.
\end{itemize}

\subsection{Acceptance}

By signing this Agreement, both Parties confirm their understanding and acceptance of the terms and commit to fulfilling their responsibilities to achieve successful project completion.

\newpage

\vspace{0.5cm}

\textbf{Signed:}

\vspace{0.5cm}

\textbf{Sundeep Kumar} (Team Member): 

\includegraphics[width=2cm]{dev_signature.jpg}

\vspace{0.5cm}

\textbf{Kamran Younis} (RP): 

\includegraphics[height=1cm]{rp_signature.jpg}

\vspace{0.5cm}

\textbf{Date:} 2 November 2025

\newpage

\tableofcontents

\newpage

\section{Introduction}

The \textbf{Smart Complaint and Maintenance Management System (SCMMS)} is a digital platform developed for Namal University to efficiently manage maintenance-related issues. It replaces the manual reporting process with a structured, trackable, and transparent system, allowing students, faculty, and staff to report, monitor, and resolve campus facility problems seamlessly. This system enhances accountability, reduces delays, and ensures timely maintenance of university infrastructure.

\section{Problem Statement}

Currently, the process of handling maintenance requests at Namal University is informal and inefficient. Complaints are often communicated verbally or through messages without any systematic tracking or record-keeping. This leads to delays in issue resolution, duplication of efforts, and lack of accountability. A centralized, online complaint management system is therefore required to enhance efficiency, transparency, and user satisfaction.

\section{Project Objective}

\begin{enumerate}
    \item Provide an online portal for submitting and tracking maintenance complaints.
    \item Enable categorized complaints (e.g., Electrical, Plumbing, IT, Furniture).
    \item Allow administrators to assign complaints to maintenance staff and update status.
    \item Provide real-time updates to users on complaint progress.
    \item Generate summary reports and insights on complaint resolution performance.
\end{enumerate}

\section{Stakeholder Identification}

\begin{table}[h]
\centering
\begin{tabularx}{\textwidth}{|l|X|}
\hline
\textbf{Stakeholder} & \textbf{Role/Description} \\
\hline
Students / Faculty / Staff & End-users who submit complaints \\
\hline
Maintenance Team & Handles and resolves assigned issues \\
\hline
Admin / Supervisor & Monitors complaints and assigns staff \\
\hline
Requirement Provider (RP) & Oversees progress, provides feedback, and approves milestones \\
\hline
\end{tabularx}
\end{table}

\section{Software Development Methodology}

The \textbf{Kanban Framework} is selected for developing the \textit{Namal Complaint and Maintenance Management System}. Kanban focuses on continuous delivery, workflow visualization, and adaptability --- which align closely with this project's nature. Since complaint management involves evolving requirements, ongoing updates, and regular interaction with the Requirement Provider (university administration), Kanban ensures smooth progress without rigid sprint boundaries.

Unlike time-boxed Agile methods, Kanban enables flexible task prioritization. Each module --- such as login, complaint registration, dashboard, or report generation --- can move independently across stages like \textit{To Do}, \textit{In Progress}, \textit{Testing}, and \textit{Completed}. This approach supports consistent improvement, transparency in task flow, and early feedback integration.

\subsection{Justification}

\begin{itemize}
    \item Allows incremental delivery without fixed iteration cycles.
    \item Ideal for projects requiring regular Requirement Provider input and evolving priorities.
    \item Improves efficiency through visual workflow tracking and load balancing.
    \item Reduces bottlenecks and supports multiple active development tasks.
\end{itemize}

\section{Tools and Technologies}

We will use MERN stack for Namal Complaint Management System Website. Tools and Technologies are given below.

\begin{itemize}
    \item \textbf{Frontend:} React/Next.js will be used for frontend development.
    \item \textbf{Backend:} Node, Express will be used for Backend Development.
    \item \textbf{Database:} MongoDB / PostgreSQL will be used as Database.
    \item \textbf{Authentication:} JWT (JSON WEB TOKEN)
    \item \textbf{Cloud Storage:} Cloudinary
    \item \textbf{Visualization and Task Management:} Github
    \item \textbf{Version Control and Deployment:} Github and Vercel
    \item \textbf{Design and Documentation:} LaTeX and Figma
\end{itemize}

\section{References}

\begin{enumerate}
    \item React Documentation. (2024). \textit{React -- A JavaScript Library for Building User Interfaces}. Retrieved from \url{https://react.dev/}
    
    \item Express.js Foundation. (2024). \textit{Express Framework Documentation}. Retrieved from \url{https://expressjs.com/}
    
    \item MongoDB, Inc. (2024). \textit{MongoDB Official Documentation}. Retrieved from \url{https://www.mongodb.com/docs/}
    
    \item Cloudinary Ltd. (2024). \textit{Cloudinary API Documentation}. Retrieved from \url{https://cloudinary.com/documentation}
    
    \item Atlassian. (2024). \textit{Kanban Methodology Overview}. Retrieved from \url{https://www.atlassian.com/agile/kanban}
\end{enumerate}

\end{document}